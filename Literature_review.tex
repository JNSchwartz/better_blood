\section{Literature review}
The number of researches of traditional machine learning (TML) and deep learning (DL) for leucocytes classification shows exponential growth in recent times.
%KHAN, Siraj, et al. A review on traditional machine learning and deep learning models for WBCs classification in blood smear images. IEEE Access, 2020, 9: 10657-10673.
DL approaches has the advantage over TML to offer highly automated systems, because in case of DP, feature extraction is learned from data and not designed by humans, as in case of TML.
For this reason DL solutions using CNN got great attention from researchers, especially since modern CNN architectures appeared.

The 
%KHAN, Siraj, et al. A review on traditional machine learning and deep learning models for WBCs classification in blood smear images. IEEE Access, 2020, 9: 10657-10673.
survey collects 
The WBS detection process with CNN is structured from 4 main steps: preprocessing, feature extraction, feature selection and classification.
Our research is based on 
%SHAHZAD, Asim, et al. Categorizing white blood cells by utilizing deep features of proposed 4B-AdditionNet-based CNN network with ant colony optimization. Complex & Intelligent Systems, 2022, 8.4: 3143-3159.


\subsection{}

\color{green} \end
\color{red} \end
\section{Literature review}
Until nowadays, a lot of researches has been conducted on WBC image classification, which indicates the importance of the topic.
The \cite{b7} survey gives a very good introduction and overview of the more significant researches. It presents a wide spectrum of different technics and methodologies used by researchers to create semi or fully automatic systems for WBC counting and classification. The survey lists state of the art implementations categorized by the used methodologies based on the reached accuracy.

The WBS detection process with CNN is structured from 4 main steps: preprocessing, feature extraction, feature selection and classification.

Preprocessing have the role to remove irrelevant information from the image and enhance the contrast \cite{b8}, \cite{b9}, \cite{b10}, \cite{b11}. Good preprocessing can significantly improve the accuracy of the classification \cite{b12}.

CNN can be used for classification, but also for feature extraction (FE). FE is the process of extracting relevant information from raw data. However, 
the problem of extracting appropriate features that can reflect the intrinsic content of a piece of data or dataset as complete as possible is still a challenge for most FE techniques \cite{b13}. 
Shape color and texture features can be obtained with traditional image processing approach, but they are also present in sublayers of pretained CNNs like AlexNet.
In \cite{b14} study a traditional image processing approach and CNN (FE) with AlexNet is compared. Both approaches have comparable accuracy but misclassification is 
more in case of the traditional approach compared to the CNN approach. However CNN approach require more computational resources.

Feature selection (FS) is the process of removing the redundant or not important features from the output feature set of the FE. FS is very important to achieve high accuracy and deacreased runtime. 
The selection of redundant or innapropiate features results in low accuracy. Using lots of feature increases the runtime significantly \cite{b15}, \cite{b16}.

Classification is the final and most important stage. At this level every image gets the label of that class, which has the highest probability to correspond to the image \cite{b17}. 
The probability is usually calculated by using the softmax activation function. This function returns a number between 0 and 1 \cite{b18}. 

This paragraph contains a brief overview and comparison of state of the art systems.  
In \cite{b19} a color space conversion and k-means algorithm based new WBC nucleus segmentation method has been developed. The achieved accuracy is 99.42\% on BCCD dataset.
Paper \cite{b20} investigates image transformation operations and generative adversarial networks for data augmentation and state-of-the-art deep neural networks for classification. The achieved accuracy is 98.8\% using DenseNet-169 on CIFAR-100 dataset.
In \cite{b21} research a three step method is proposed: detecting the nucleus and cytoplasm with a new algorithm, extracting features, and classification using SVM model. The achieved accuracy is 94.20\% on BCCD dataset.
In \cite{b5} research the system contains contrast-limited adaptive histogram equalization as preprocessor; ResNet50, EfficientNetB0 and 4B-AdditionNet a for feature extraction; 
ant colony optimization for feature selection and support vector machine as well as quadratic discriminant analysis for classification. The achieved accuracy is 98.44\% on Blood Cell Images dataset.

The proposed CNN architecture in this research is very similar to the \cite{b5} papers CNN arhitecture, with modifications in the feature extraction process. 
The modification includes replacing the ResNet50 to ResNet50V2. ResNet50V2 is a modified version of ResNet50. ResNet50V2 was used in different medical application with success: 
\cite{b22} for histological analysis of breast tissue, \cite{b23} for detection of autism spectrum disorders on Resting-state Functional Magnetic Resonance Images 
\cite{b6} for detecting COVID-19 and pneumonia from chest X-ray images.
